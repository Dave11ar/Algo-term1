\documentclass{article} 
\usepackage{amsfonts} 
\usepackage{amsmath,amsthm,amssymb} 
\usepackage{mathtext} 
\usepackage[english,russian]{babel} 
\usepackage{setspace} 
\usepackage{amsmath} 
\usepackage{alltt} 

\setlength{\textwidth}{7in}
\setlength{\topmargin}{-0.5in} 
\setlength{\textheight}{9.1in}
\setlength{\oddsidemargin}{-0.3in}
\setlength{\textwidth}{7in}

\begin{document} 
\noindent 
\onehalfspacing  

\section{Задача 1}
Построим дерево отрезков, в котором будем хранить две величины:
\begin{enumerate}
\item сумму элементов на отрезке, обозначим $С$
\item сумму вида $a_L + 2a_{L+1} + ... + (R-L+1)a_R$ на отрезке, обозначим $S$
\end{enumerate}
Обозначим $C_l$, $C_r$, $S_l$, $S_r$ как соответсвующие суммы левого и правого сыновей и $L, R$ как левую и правую границы соответвующего отрезка.\\
$C$ считается тривильно: $C_l + C_r$.\\
$S$ будем считать так: $S_l + (S_r + \frac{(R - L + 1)}{2}C_r)$, где $\frac{(R - L + 1)}{2}$ длина отрезка соответвующего левому сыну.\\
Изменение элементов будем производить с помощью проталкивания. Пусть $x$ --- старое значение элемента, $y$ --- новое, тогда при проталкивании изменяем вершину соответсвующим образом $C=C+(y-x)$, $S=S+p\cdot(y-x)$, где $p$ --- позиция элемента в отрезке соответвующем вершине.\\
Будем считать ответ на запрос $L$, $R$ так же как пересчитывали $S$ --- суммируем по всему отрезку с $L$ по $R$ соответсвующие $S + p \cdot C$, где $p$ --- левая граница искомого отрезка минус $L$.

\section{Задача 2}
Отсортируем прямоугольники по $x$. Сожмем координаты $y$ (отсортируем и сопоставим каждой координате ее номер в отсортированном массиве) и построим по ним дерево отрезков в вершинах которого будем хранить пару значений --- максимальное количество пересеченных прямоугольников и координату $y$. Пересчитывание очевидно --- берем максимум из двух сыновей и соответсвующую координату.\\
Из каждого прямоугольника сделаем $2$ события --- открытие прямоугольника(его левая граница) и закрытие(правая граница). Будем идти слева направо по этим событиям, и если встречаем открытие то делаем прибавление единицы на отрезке соответвующем верхней и нижней границам прямоугольника, соответвенно если встречаем закрытие, то убавляем единицу на том же отрезке.\\
Максимум пересеченных прямоугольников для точки с координатой $x$ это значение в корне ДО на итерации соответвующей $x$. Тогда взяв максимум из значений корня на всех итерациях получим максимальное количество прямоугольников в пересечении. Ответом на задачу будет точка с координатой $x$ --- координата соответствующая итерации на которой был получен максимум, $y$ --- координата которая была записана в паре с тем максимумом. \\
Ассимптотика --- $2nlog_{2}{2n}=O(nlogn)$

\section{Задача 4}
Значение $S$ каждый раз зануляется если $S+a_i$ меньше нуля. Заметим что искомая величина --- сумма элементов после последнего раза когда $S$ был равен нулю.\\ 
Докажем, что это самый правый максимальный суффикс, для этого достаточно доказать на этом суффиксе нету ни одного зануления, и что на следующей позиции слева от суффикса происходит зануление.
\begin{enumerate}
\item пусть справа от левой границы суффикса есть зануление, тогда сумма от левой границы до этого зануления $\leq$ $0$, и ее нужно выкинуть из суффикса тк либо он не максимальный, если меньше 0, либо не самый правый из максимальных, если равно 0.
\item пусть сразу слева от суффикса не происходит зануление, тогда сумма от ближайшего слева зануления до суффикса $> 0$ $\Rightarrow$ наш исходный суффикс не максимален, тк мы можем прибавить эту сумму к нему.
\end{enumerate}
Свели задачу к поиску максимального суффикса. Решим ее деревом отрезков: будем хранить в листах сумму соответсвующего суффикса($d.size$ листов). В вершинах будем хранить максимум среди суффиксов(пересчет очевиден --- берем максимум из сыновей).\\
При изменении элемента $d$ прибавим разность между его новым и старым значением к префиксу ДО кооторому соответвуют суффиксы исходного массива в которые входит $d$. \\
Ответом на запросы будет значение в корне --- максимальный суффикс.
\end{document}
