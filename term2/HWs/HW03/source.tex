\documentclass{article} 
\usepackage{amsfonts} 
\usepackage{amsmath,amsthm,amssymb} 
\usepackage{mathtext} 
\usepackage[english,russian]{babel} 
\usepackage{setspace} 
\usepackage{amsmath} 
\usepackage{alltt} 

\setlength{\textwidth}{7in}
\setlength{\topmargin}{-0.5in} 
\setlength{\textheight}{9.1in}
\setlength{\oddsidemargin}{-0.3in}
\setlength{\textwidth}{7in}

\begin{document} 
\noindent 
\onehalfspacing  

\section{Задача 1}
Заведем два массива --- $sub$ и $pref$, будем обрабатывать данные нам запросы. Пусть нам даны $k, x, l, r$, совершим следующие операции:
\begin{enumerate}
\item
$pref[l]+=x$
\item
$pref[l+1]+=-x+k$
\item
$pref[r+1]+=-x-k(r-l+1)$
\item
$pref[r+2]+=x+k(r-l)$
\end{enumerate}
Заполним массив $sub$ префиксными суммами массива $pref$, тогда очевидно $sub[l] = x$, $sub[l + 1...r]=k$, $sub[r+1]=-x-k(r-l)$, $sub[r+2]=0$. Легко заметить что $sub$ это разность соседних элементов массива $a$ к которому делаем прибавление. Сделав много операций прибавления прогрессии этот инвариант сохранится тк они не влияют друг на друга.\\
Насчитаем префиксные суммы массива $sub$, они будут выглядеть так --- $pref[l]=x$, $pref[l+1 < i \leq r]=x+k(i-l)$, $pref[r+1...n]=0$, тогда очевидно в массиве префиксных сумм будут лежать значения массива которые нужно прибавить к элементам массива $a$ чтобы получить нужную нам последовательность после всех модификаций.\\
Каждый запрос выполняется за 4 прибавления в массиве те $O(1)$, запросов $m$, за $O(n)$ насчитаем префиксные суммы, итоговая ассимптотика --- $O(n+m)$.
\section{Задача 4}
Аналогична пятой только без снятия пометки.
\section{Задача 5}
Построим на дереве hld, на каждом пути постоим дерево отрезков на самую высокую непомеченную вершину и самую низкую непомеченную вершину.\\
Если нам пришел запрос снять пометку или поставить пометку, то соответствующим образом изменим ДО.\\
Если нам пришел запрос ответа от двух вершин, то найдем их lca за $O(logn)$ подъемом по hld, сделаем запрос в ДО на котором лежит lca за $O(logn)$ --- самая низкая непомеченная на отрезке от lca до корня пути, если такая существует то это и есть ответ, иначе пойдем по путям наверх и на каждом пути будем смотреть в корень дерева, если самая высокая непомеченная вершина существует и она выше вершины пути в котором мы стоим, то ответ содержится в этом дереве и мы найдем его запросом на самую низкую вершину на пути от вершины в которой мы стоим до корня пути, иначе свободной вершины на нужной нам части пути нет и мы пойдем вверх к корню.\\
Тогда итоговая ассимтотика $O(logn)$, тк мы найдем lca за $O(logn)$, поднимемся в hld за $O(logn)$, и сделаем запрос в двух ДО(в которой лежит lca и в которой существует непомеченная вершина на части пути которая ведет к корню дерева) за $O(logn)$, предподсчет за $O(nlogn)$ тк строим hld.

\end{document}
