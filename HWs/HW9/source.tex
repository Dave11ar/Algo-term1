\documentclass{article} 
\usepackage{amsfonts} 
\usepackage{amsmath,amsthm,amssymb} 
\usepackage{mathtext} 
\usepackage[english,russian]{babel} 
\usepackage{setspace} 
\usepackage{amsmath} 
\usepackage{alltt} 
\begin{document} 
\noindent 
\large 
\onehalfspacing  
\section{Задача 1}
Пусть для определенности исходный массив --- $a$. Заведем 2 массива --- $dp$ и $last$ длины $n$ и будем считать результат на префиксе, тогда ответ будет лежать в $dp[n]$. Пусть $last[a[1]]=1$ и $dp[1] = 1$, так как последовательность на префиксе длины $1$ всего одна. Будем пересчитывать значения так:\newline
Идем от $i = 2$ до $n$, в массиве $last[k]$ будем хранить последнее вхождение числа $k$ на префиксе.\newline
Есть 2 случая:\newline
\begin{enumerate}
\item Число $a[i]$ мы встретили впервые. Тогда количество подпоследовательностей на префиксе будет --- $2\cdot dp[i-1] + 1$, так как мы можем либо взять подпоследовательности из $dp[i-1]$ без изменения, либо дописать к ним в конец $a[i]$, либо взять само $a[i]$ как подпоследовательность.
\item У нас уже были вхождения числа $a[i]$. Тогда число $a[i]$ как подпоследовательность мы брать не должны, так как оно уже было посчитано. Присвоим $dp[i]$ значение $2\cdot dp[i-1]$, в нем некоторые подпоследовательности будут посчитаны дважды: а именно подпоследовательности $dp[last[a[i]] - 1]$, так как приписывая им в конец $a[last[a[i]]]$ или $a[i]$ мы получаем одинаковые подпоследовательности, так как $a[i] = a[last[a[i]]]$. Тогда функция для $dp[i]$ такая:\newline
\begin{equation*}
dp[i] =
\begin{cases}
2\cdot dp[i - 1] + 1,\; last[a[i]] = -1\\
2\cdot dp[i - 1] - dp[last[a[i]] - 1],\; last[a[i]] \neq -1
\end{cases}
\end{equation*}
Ответом будем $dp[n]$.
\end{enumerate}
\section{Задача 2}
Докажем в обе стороны:\newline
\begin{enumerate}
\item Длина максимального подпалиндрома не меньше НОП:\newline
Предположим обратное: пусть длина максимального подпалиндрома меньше НОП. Пусть $ind_c$ --- массив индексов НОП в текущем массиве, $ind_r$ --- массив индексов НОП элементов развернутого массива в исходном, $m$ --- длинна НОП. Тогда рассмотрим 3 случая:\newline
\begin{enumerate}
\item Все элементы $ind_r$ находятся правее $ind_c$, тогда элементы исходного массива с индексами \newline $ind_c[1]... ind_c[m], ind_r[1]...ind_r[m]$ --- подпалиндром длины $2m$.
\item Все элементы $ind_r$ находятся левее $ind_c$, аналогично элементы исходного массива с индексами \newline $ind_r[1]... ind_r[m], ind_c[1]...ind_c[m]$ --- подпалиндром длины $2m$.
\item Массивы $ind_c$ и $ind_r$ пересекаются. Тогда, если $ind_r[1]$ находится правее первой половины $ind_c$, то мы сможем составить подпалиндром длины $m$ из элементов $ind_c[1]...ind_c[\frac{m}{2}],ind_r[\frac{m}{2} + 1]...ind_r[m]$, если же он не правее первой половины, то в $ind_r$ будем идти до $\frac{m}{2}$ и искать элемент который правее первой половины и аналогично строить подпалиндром.\newline
Если дошли до $\frac{m}{2}$ элемента массива и он все еще стоит не правее половины элементов массива, то мы может построить подпалиндром длины $m$ так --- берем элементы исходного массива \newline $ind_r[1]...ind_r[\frac{m}{2}],ind_c[\frac{m}{2} + 1]...ind_c[m]$.\newline
Таким образом доказали, что длина максимального подпалиндрома не меньше длины НОП.
\end{enumerate}
\item Длина максимального подпалиндрома не больше НОП:\newline
Предположим обратное: пусть длина максимального подпалиндрома больше нашей НОП, но  в силу того, что палиндром равен самому себе развернутому, тогда он будет как в $a$, так и в $a^r$, а значит будем общей подпоследовательностью обоих массивов большей данной нам НОП --- противоречие.
\end{enumerate}
Доказали, что длина максимального подпалиндрома не больше $m$ и не меньше $m$ $\implies$ длина максимального подпалиндрома равна $m$.
\section{Задача 3}
Пусть для определенности исходный массив --- $a$. Заведем массивы $dp_1$, $dp_2$, $ind_1$ и $ind_2$ длины n. Будем считать НВП в $dp_1$ и в $ind_1[i]$ записывать индекс куда в $dp_1$ был записан элемент исходного массива на своей итерации. Посчитаем в $dp_2$ наибольшую убывающую подпоследовательность в перевернутом исходном массиве, и аналогично посчитаем $ind_2$, в индексах исходного массива, то есть на итерации $i$ будем писать в $ind_2[n-i+1]$. Создадим массив $res[1...n]$ и заполним его нулями. Будем идти от $i=1$ до $n$, если $ind_1[i] + ind_2[i] - 1$ равно длине НВП исходного массива, то $a[i]$ содержится в одной из НВП, так как у $a[i]$ есть $ind_1[i] - 1$ отсортированных элементов слева, и $ind_2[i] - 1$ отсортированных элементов справа. Тогда элементы у которых $ind_1[i] + ind_2[i] - 1$ не равно длине НВП --- не содержатся ни в одной НВП, если у элемента выполняется данное равенство --- запишем $res[ind_1[i]]++$. Повторно пройдем по массиву и для элементов у которых выполняется равенство будем смотерть на значение $res[ind_1[i]]$, если оно равно $1$, то на $ind_1[i]$ месте в НВП может стоять только этот элемент, а значит --- он входит во все НВП, если же $res[ind_1[i]]>1$, то элемент будет входить хотя бы в одну НВП. 
\end{document}

