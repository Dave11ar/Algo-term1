\documentclass{article} 
\usepackage{amsfonts} 
\usepackage{amsmath,amsthm,amssymb} 
\usepackage{mathtext} 
\usepackage[english,russian]{babel} 
\usepackage{setspace} 
\usepackage{amsmath} 
\usepackage{alltt} 
\begin{document} 
\noindent 
\large 
\onehalfspacing  
\section{Задача 1}
Пусть для определенности первая куча записана в массиве $a$. Положим во вторую кучу пару: значение корня первой кучи и ее индекс в массиве $a$, а так же ее детей с соответствующими индексами. Вторая куча --- куча по первому элементу пары. Пусть корень второй кучи --- пара значение $b$, $i$. Теперь совершим $k-1$ таких действий: будем класть во вторую кучу сыновей вершины $i$ из первой кучи, а затем удалять корень второй кучи. Тогда после $k-1$ итерации мы удалим $k-1$ минимальных элементов, и значение корня второй кучи и будет $k$-й порядковой статистикой.\newline
Докажем корректность: мы хотим, чтобы на $i$-й итерации мы удаляли $i$-й минимальный элемент. Докажем по индукции: на первой итерации все верно, так как мы удалим корень, рассмотрим $i$-ю итерацию: пусть до нее были удалены $i-1$ минимумов, тогда был удален и ее отец, так как он меньше нее, а значит она была добавлена в кучу и будет удалена на $i$-й итерации $\implies$ алгоритм корректен.
\section{Задача 2}
Будем хранить в каждой вершине переменную int $change$, которая сначала будет равна нулю. При совершении операции $changeKeys$ $h$ $x$ будем прибавлять к $change$ корня значение $x$ --- это очевидно работает за $O(1)$.\newline
Создадим функцию $push$, которая будем принимать вершину. Она устроена таким образом --- прибавим к $change$ сынов вершины наше значение $change$, прибавим самому значению вершины $change$ и обнулим $change$ у вершины.\newline
Теперь в остальных функциях перед тем как будем обращаться к элементу будем делать от него $push$ --- это не изменит ассимптотику, так как $push$ работает за $O(1)$.\newline 
Докажем корректность: во всех функциях мы обращаемся к вершинам кучи последовательно, тоесть мы не можем обратиться к сыну вершины не обратившись к самой вершине --- тогда за счет операции $push$ значение в вершине к которой мы обратимся всегда будет верно, а прибавление на потомках произойдет, так как мы протолкнули наш $change$.
\section{Задача 3}
Пусть для определенности нам дан массив $a$ длинны $n$. Найдем $\frac{n}{2}$-ю порядковую статистику массиве $a$ за $O(n)$. Постоим две кучи за $O(n)$: в первую поместим нашу порядковую статистику и все элементы левее(меньше него) --- это обратая куча(элемент в корне больше всех элементов поддерева), а во вторую все элементы правее(больше него). Тогда в первой куче у нас $\frac{n}{2}$ минимальных элементов, а во второй куче --- $\frac{n}{2}$ максимальных. Наши функции будут работать так:
\begin{enumerate}
\item $medianElement$\newline
Наша медиана --- $\frac{n}{2}$-я порядковая статистика --- корень первой кучи, просто вернем его. Работает за $O(1)$.
\item $deleteMedian$\newline
Наша медиана --- корень первой кучи. Удалим корень первой кучи и сольем ее сыновей. Это работает за $O(\log{n})$. Теперь у нас есть два варианта:
\begin{enumerate}
\item Количество элементов стало четно и новая медиана лежит во второй куче --- положим корень второй кучи в первую($O(\log{n})$), удалим его из второй и сольем ее сыновей($O(\log{n})$).
\item Если количество элементов стало нечетно и новая медиана лежит в первой куче --- ничего не будем делать.
\end{enumerate}
Общая ассимптотика --- $O(\log{n})$.
\item $insert$ $x$\newline
Поймем в какую кучу надо добавлять элемент:
\begin{enumerate}
\item Если $x$ меньше либо равно корня первой кучи, добавим его в первую кучу и:
\begin{enumerate}
\item Если количество элементов стало четно и, следовательно, медиана лежит в первой куче --- ничего не будем делать
\item Если количество элементов стало нечетно и, следовательно, медиана лежит в первой куче, но она не корень --- положим корень первой кучи во вторую($O(\log{n})$), удалим его из первой и сольем его детей($O(\log{n})$).
\end{enumerate}
\item Если $x$ больше корня первой кучи, добавим его во вторую кучу и:
\begin{enumerate}
\item Если количество элементов стало четно и, следовательно, медиана лежит во второй куче --- положим корень второй кучи в первую($O(\log{n})$), удалим его из второй и сольем его детей($O(\log{n})$).
\item Если количество элементов стало нечетно и, следовательно, медиана лежит в первой куче --- ничего не будем делать.
\end{enumerate}
\end{enumerate}
\end{enumerate}
\end{document}