\documentclass{article} 
\usepackage{amsfonts} 
\usepackage{amsmath,amsthm,amssymb} 
\usepackage{mathtext} 
\usepackage[english,russian]{babel} 
\usepackage{setspace} 
\usepackage{amsmath} 
\usepackage{alltt} 

\setlength{\textwidth}{7in}
\setlength{\topmargin}{-0.5in} 
\setlength{\textheight}{9.1in}
\setlength{\oddsidemargin}{-0.3in}
\setlength{\textwidth}{7in}

\begin{document} 
\noindent 
\onehalfspacing  
\section{Задача 3}
Будем набирать предметы в подходы последовательно. Создадим массив $dp$ размера $2^n$. В каждом $dp[i]$ будем хранить пару значений: количество набранных рюкзаков и оставшееся место в последнем рюкзаке(рюкзаком будем называть набор набранных вещи в подход). Будем говорить что что пара $i_1, j_1$ оптимальнее пары $i_2, j_2$ если:
\begin{enumerate}
\item $i_1 < i_2$(количество набранных рюкзаков для данной маски предметов меньше $\implies$ набор оптимальнее).
\item $i_1 = i_2$ и $j_1 < j_2$(количество набранных рюкзаков равно, но в последнем осталось больше места $\implies$ набор оптимальнее).
\end{enumerate}
В остальных случаях второй набор оптимальнее.\\
Будем пересчитывать динамику так: перебираем подмаску, и для каждой подмаски пытаемся добавить в нее каждый возможный предмет, если такое $dp$ уже насчитано, то возьмем более оптимальное, как описано выше. Ответом будет количество рюкзаков в $dp[2^{n}]$.\\
Наш алгоритм найдет оптимальное разбиение, так как для каждого $dp$ все его подмаски насчитаны оптимально, предположим что на каком то шаге мы можем как то поменять порядок предметов между рюкзаками и положить последний элемент не в последний рюкзак, но тогда такой случай был учтен когда мы считали $dp$ по маске предметов в этом рюкзаке.\\
Алгоритм отработает за $\mathcal{O}(2^{n}n)$, так как мы считаем $2^n$ $dp$ и для каждого перебираем $n$ предметов которые пытаемся добавить.
\section{Задача 4}
Докажем первое утверждение по индукции: пусть на обоих подеревьях выполенено условие $\sum\limits_{i=1}^{m}2^{-d_i}\leq 1$, тогда посчитаем сумму для нашей вершины, она равна сумме результатов на сыновьях деленной на $2$, так как для каждой вершины ее глубина увеличилась на $1$ $\implies$ каждое слагаемое вида $2^{-d_i}$ превратилось в $2^{-(d_i+1)}=2^{-d_i-1}=\frac{1}{2}2^{-d_i}$\\
Критерий для $\sum\limits_{i=1}^{m}2^{-d_i}= 1$ --- это то, что дерево полное(каждая вершина имеет $0$ или $2$ ребенка). Докажем это: чтобы сумма была равно $1$ нам необходимо, чтобы ответы на сыновьях были так же равны $1$, если какая то вершина имеет одного сына, то ее максимальная сумма будет равна $\frac{1}{2}$, так как у единственного ребенка максимальная сумма $1$. Тогда если хотя бы у одной вершины сумма не равна $1$, то у всех ее предков(а значит и у корня) она будет меньше $1$.
\section{Задача 5}
Рассмотрим начальную($v_1$) и конечную($v_k$) вершины, пусть между ними есть путь $S$. Тогда разобьем вершины на внутренние вершины и вершины пути. Внутренние --- все вершины правых поддеревьев вершин пути левее $lca$, и аналогично все вершины левых поддеревьев вершин пути правее $lca$. Внутренние вершины будут точно посещены, потому, что все они больше $v_1$ и меньше $v_k$. Тогда заметим, что внутренние вершины мы посетим не более $3$-х раз:
\begin{enumerate}
\item На пути в левое поддерево во время поиска другой вершины.
\item Когда ищем саму эту вершину.
\item На пути из правого поддерева --- выше.
\end{enumerate}
Таких посещений будет не больше $3k = \mathcal{O}(k)$.\\
Вершины пути же могут и не входить в множество нужных нам вершин, мы посетим каждую аналогично не более $3$-х раз, таких вершин не более $2\log_2{n}$, тогда посещений их не более $3\cdot 2\log_2{n} = \mathcal{O}(\log_2{n})$.\\
Итоговая ассимптотика $\mathcal{O}(\log_2{n}+k)$.

\end{document}
